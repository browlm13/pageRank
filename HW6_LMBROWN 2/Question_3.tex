\title{Naive Bayes Classification}
\author{LJ Brown}
\documentclass{article}
\usepackage{amsmath}
\usepackage[utf8]{inputenc}
\begin{document}
\maketitle

% new commands:
\newcommand{\A}{x_1 \, and \, \dots \, x_n}

% Hide Section Numbers
\makeatletter
% \@seccntformat is the command that adds the number to section titles
% we make it a no-op
\renewcommand{\@seccntformat}[1]{}
\makeatother

\section{Homework 6, Question 3}\label{abstract}

% Subject
You need to categorize vehicles into the following categories: truck, suv, and sedan using 5 features. Perform a naïve Bayes classification with the following probabilities. Show your work.
	 	 

% 
% Table 1
%

% Title
\begin{center} Table 1 \end{center}

\begin{center}
%\begin{table}
 \begin{tabular}{||c c c c ||} 
 \hline
 $c$ & TRUCK & SUV & SEDAN \\ [0.5ex] 
 \hline\hline
 $P(c)$&0.35&0.4&0.25\\
 \hline
 $P(f_1|c)$&0.2&0.01&0.2\\
 \hline
 $P(f_2|c)$&0.01&0.1&0.05\\
 \hline
 $P(f_3|c)$&0.1&0.001&0.005\\
 \hline
 $P(f_4|c)$&0.001&0.2&0.005\\
 \hline
 $P(f_5|c)$ & 0.005& 0.008 & 0.01\\ [1ex] 
 \hline
\end{tabular}
%\caption{Given Homework 5 Question 5 Table}
%\label{table:kysymys}
%\end{table}
\end{center}



%
%	Scratch
%
%The columns are mutually exclusive events, or c is a categorical variable
% assume features are independent variables
% features are dependent on c tho as was play



% new commands
\newcommand{\agb}[2]{P(#1|#2)}
\newcommand{\bayes}[2]{\agb{#1}{#2} = \frac{P(#1) \agb{#2}{#1}}{P(#2)}}


%
% Bayes' Theorem
%

% reference equation number \ref{Bayes' Theorem}

% Title
\begin{center} Bayes' Theorem \end{center}

% Equation
\begin{equation} %\tag{1}
		\nonumber \bayes{A}{B}
		\label{Bayes' Theorem}
\end{equation} \\


%
%	Classifying c given a feature set
%

\section{Classifying c given a set of features}


%
%	Variable Definitions
%

$F$ is a set of features  \\
$f_i$ is a member of $F$ \\ \\
$C = \{ \text{TRUCK, SUV, SEDAN} \} $ \\ 
$c_i$ is a member of $C$ \\

%
%	Searching for
%

For a given feature set, $F$, I classify c as,
\begin{equation}
	\nonumber
	%
	c =
	%
 	\max_{c_i \in C} \big( \agb{c_i}{F} \big)
	%
\end{equation}

%
% Bayes Theorem rewritten for a given set of features
%

Rewriting Bayes Theorem for a given class, $c_i$, and feature set, $F$,

\begin{equation}
	\nonumber \agb{c_i}{F} = \frac{P(c_i) \agb{F}{c_i}}{P(F)}
\end{equation}

% expanded equation above (1)
%\begin{equation} 
%	\nonumber \agb{c}{f_1 \cap f_2 \cap f_3} = \frac{P(c) \agb{f_1}{c}\agb{f_2}{c}\agb{f_3}{c}}{P(f_1)P(f_2)P(f_3)}
%\end{equation}

% expanded equation above (1)
\begin{equation}
	\nonumber \agb{c_i}{F} = 
	%
	\frac{P(c_i) \agb{F}{c_i}}{P(F)} =
	%
	P(c_i) \prod\limits_{f_i \in F} \frac{ \agb{f_i}{c_i}}{P(f_i)}
	%
\end{equation}

%
%	When classifying c I think you can drop the denominator and just compare relative scores
%

When classifying c I think you can drop the denominator and just compare relative scores. I think this is fine because the events in C are mutually exclusive.

\begin{equation}
	\nonumber
	%
	P(c_i) \prod\limits_{f_i \in F} \frac{ \agb{f_i}{c_i}}{P(f_i)}
	%
	\propto
	%
	P(c_i) \prod\limits_{f_i \in F} \agb{f_i}{c_i}
\end{equation}

\begin{equation}
	\nonumber
	%
	\agb{c_i}{F}
	%
	\propto
	%
	P(c_i) \prod\limits_{f_i \in F} \agb{f_i}{c_i}
\end{equation}


% porportional max relation

\begin{equation}
	\nonumber
	%
 	\max_{c_i \in C} \big( \agb{c_i}{F} \big)
	%
	=
	%
	\max_{c_i \in C} \Big( P(c_i) \prod\limits_{f_i \in F} \agb{f_i}{c_i} \Big)
\end{equation}

%
% Final equation
%

So the final equation the program uses to classify is,
\begin{equation}
	\nonumber
	%
 	c
	%
	=
	%
	\max_{c_i \in C} \Big( P(c_i) \prod\limits_{f_i \in F} \agb{f_i}{c_i} \Big)
\end{equation}

\begin{equation}
	\nonumber
	%
	\text{Where $C$ and $F$ are given as parameters.}
\end{equation}

%
%	Answers for Home work
%

\section{Question 3 Answers}

\begin{center} Question 3.a \end{center}

a. 	What category would you assign to the vehicle (f1, f2, f3)? \\ \\

$F = \{ \text{f1, f2, f3} \}$

\begin{equation}
	\nonumber
	%
 	c
	%
	=
	\max_{c_i \in C} 
	% 
	\Big(
	%
	P(c_i) \agb{f_1}{c}\agb{f_2}{c_i}\agb{f_3}{c_i}
	%
	\Big)
\end{equation}

example calculation (happens to be the maximum) for a single value of $c_i$,

\begin{equation}
	\nonumber 
	%
	\agb{TRUCK}{F} \approx
	%
	0.35 * 0.2 * 0.01 * 0.1 
	%
	= 0.00007
\end{equation}

\begin{equation}
	\nonumber
	%
 	c
	%
	=
	\max_{c_i \in C} 
	% 
	\Big(
	%
	P(c_i) \agb{f_1}{c}\agb{f_2}{c_i}\agb{f_3}{c_i}
	%
	\Big)
	%
	= \, \text{TRUCK}
\end{equation}


Answer for question 3.a is TRUCK.

\begin{center} Question 3.b \end{center}

b. 	What category would you assign to the vehicle (f1, f2, f4, f5)? \\

$F = \{ \text{f1, f2, f4, f5} \}$

\begin{equation}
	\nonumber
	%
 	c
	%
	=
	\max_{c_i \in C} 
	% 
	\Big(
	%
	P(c_i) \agb{f_1}{c}\agb{f_2}{c_i}\agb{f_4}{c_i}\agb{f_5}{c_i}
	%
	\Big)
	%
	= \, \text{SUV}
\end{equation}

Answer for question 3.b is SUV.








%
%	Scratch
%



\section{Scratch}
Below I'm trying to solve for $\agb{c_i}{F}$, not sure if it works everytime. \\ \\
$F$ is a set of features  \\
$f_i$ is a member of $F$ \\
$c_i$ is a member of $C =  \{ \text{TRUCK, SUV, SEDAN} \}$ \\

Ok so I'm assuming you can solve for $P(f_i)$ like this,

\begin{equation}
	\nonumber P(f_i) = \sum_{c_i \, \in \, C}{\agb{f_i}{c_i}}
\end{equation}

%
% Table 2
%
% Title
%\begin{center} Table 1 \end{center}

\begin{center}
%\begin{table}
 \begin{tabular}{||c c ||} 
 \hline
 $P(f_1)$&0.41\\
 \hline
 $P(f_2)$&0.16\\
 \hline
 $P(f_3)$&0.106\\
 \hline
 $P(f_4)$&0.206\\
 \hline
 $P(f_5)$ & 0.023\\ [1ex] 
 \hline
\end{tabular}
%\caption{Given Homework 5 Question 5 Table}
%\label{table:kysymys}
%\end{table}
\end{center}



\begin{center} Question 3.a \end{center}

a. 	What category would you assign to the vehicle (f1, f2, f3)?

\begin{equation}
	\nonumber \agb{c}{f_1 \cap f_2 \cap f_3} = \frac{P(c) \agb{f_1}{c}\agb{f_2}{c}\agb{f_3}{c}}{P(f_1)P(f_2)P(f_3)}
\end{equation}

\begin{equation}
	\nonumber \agb{\text{TRUCK}}{f_1 \cap f_2 \cap f_3} = \frac{0.35 * 0.2 * 0.01 * 0.1 }{0.41 * 0.16 * 0.106} \approx 0.0101
\end{equation}

\begin{equation}
	\nonumber \agb{\text{SUV}}{f_1 \cap f_2 \cap f_3} \approx 0.0005
\end{equation}

\begin{equation}
	\nonumber \agb{\text{SEDAN}}{f_1 \cap f_2 \cap f_3} \approx 0.0018
\end{equation}

This gives the same answer for question 3.b: TRUCK.

\begin{center} Question 3.b \end{center}

b. 	What category would you assign to the vehicle (f1, f2, f4, f5)?

\begin{equation}
	\nonumber \agb{\text{TRUCK}}{f_1 \cap f_2 \cap f_4 \cap f_5} \approx 0.0001
\end{equation}

\begin{equation}
	\nonumber \agb{\text{SUV}}{f_1 \cap f_2 \cap f_4 \cap f_5} \approx 0.0021
\end{equation}

\begin{equation}
	\nonumber \agb{\text{SEDAN}}{f_1 \cap f_2 \cap f_4 \cap f_5} \approx 0.0004
\end{equation}

This gives the same answer for question 3.b: SUV.

\end{document}
