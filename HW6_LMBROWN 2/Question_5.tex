\title{Naive Bayes Classification}
\author{LJ Brown}
\documentclass{article}
\usepackage{amsmath}
\usepackage[utf8]{inputenc}
\begin{document}
\maketitle

% new commands:
\newcommand{\A}{x_1 \, and \, \dots \, x_n}

% Hide Section Numbers
\makeatletter
% \@seccntformat is the command that adds the number to section titles
% we make it a no-op
\renewcommand{\@seccntformat}[1]{}
\makeatother

\section{Homework 5, Question 5}\label{abstract}

% Subject
	 Derivation of Bayes' Theorem from Kolmogorov's definition of Conditional Probability, and using it to guess the value of the missing entry ($?$) in Table 1.
	 
%	Derivation of Bayes' Theorem from Kolmogorov's definition of Conditional Probability, and it's application for guessing the value of the missing entry ($?$) in Table 1. Also the answer to question 5. %\ref{table:kysymys}.

% 
% Table 1
%

% Title
\begin{center} Table 1 \end{center}

\begin{center}
%\begin{table}
 \begin{tabular}{||c c c c c||} 
 \hline
 Weather & Temp & Humidity & Windy & Play \\ [0.5ex] 
 \hline\hline
 Rainy&Cool&Normal&FALSE&yes\\
 \hline
 Rainy&Cool&Normal&TRUE&No\\
 \hline
 Overcast&Hot&High&FALSE&Yes\\
 \hline
 Sunny&Mild&High&FALSE&No\\
 \hline
 Rainy&Cool&Normal&FALSE&Yes\\
 \hline
 Sunny&Cool&Normal&FALSE&Yes\\
 \hline
 Rainy&Cool&Normal&FALSE&Yes\\
 \hline
 Sunny&Hot&Normal&FALSE&Yes\\
 \hline
 Overcast&Mild&High&TRUE&Yes\\
 \hline
 Sunny&Mild&High&TRUE&No\\
 \hline
 Sunny & Cool& High & False	& ? \\ [1ex] 
 \hline
\end{tabular}
%\caption{Given Homework 5 Question 5 Table}
%\label{table:kysymys}
%\end{table}
\end{center}

%
% Conditional Probability (Kolmogorov's probability theory); note: P(B)≠0
%

% reference equation number \ref{Conditional Probability Definition}

% Title
\begin{center} Conditional Probability Definition \footnote{Conditional Probability definition from Kolmogorov's probability theory. } \end{center}

% Equation
\begin{equation} \tag{1}
		\nonumber P \left( A | B \right) = \frac{ P  \left( A \cap B \right)}{P \left( B \right)}
		\label{Conditional Probability Definition}
\end{equation} 

%
% Bayes' Theorem
%

% reference equation number \ref{Bayes' Theorem}

% Title
\begin{center} Bayes' Theorem \end{center}

% Equation
\begin{equation} \tag{2}
		\nonumber P \left( A | B \right) = \frac{ P  \left( A  \right) P \left( B | A \right)}{P \left( B \right)}
		\label{Bayes' Theorem}
\end{equation} \\

%
% limits and application
%

\section{Limitations}
Assumes: "Weather", "Temp", "Humidity", and "Windy" are independent categorical variables.

%The Naive Bayes Classifier outlined bellow can only be applied to datasets where the column values are independent categorical variables.


%
% Derivation of Bayes' Theorem From Conditional Probability
%

\section{Derivation of Bayes' Theorem }
Bayes' Theorem can be derived from the definition of Conditional Probability in  equation \ref{Conditional Probability Definition} by first substituting $A = B$ and $B = A$,

%Bayes' Theorem can be derived from the definition of Conditional Probability in  equation \ref{Conditional Probability Definition} by substituting $A = B$ and $B = A$, solving for $P  \left( A \cap B \right) $ and then substituting that back into the original definition.

%The definition for Conditional Probability in equation \ref{Conditional Probability Definition} after substituting $A = B$ and $B = A$ is:

% Conditional Probability Definition with A = B, B = A substitution

% Equation
\begin{equation} \tag{eq \ref{Conditional Probability Definition}: A = B, B = A substitution}
		\nonumber P \left( B | A \right) = \frac{ P  \left( B \cap A \right)}{P \left( A \right)}
\end{equation} 


This equation can be solved for $P  \left( A \cap B \right) $:

% Conditional Probability Definition with A = B, B = A solved for P(A and B)
\begin{equation} 
		\nonumber P  \left( B \cap A \right)  =  P  \left( A \right) P \left( B | A \right)
\end{equation} 

% Commutative Property of Intersection 
\begin{equation} 
		\nonumber P  \left( A \cap B \right)  =  \nonumber P  \left( B \cap A \right) 
\end{equation} 

% Solving for P(A and B) in above equation
\begin{equation} \tag{*}
		\nonumber P  \left( A \cap B \right)  =  P  \left( A \right) P \left( B | A \right)
		\label{equation:*}
\end{equation} 

% Substitution of new definition of P(A and B) into original Conditional Probability Definition 
% Obtating bayes' theorem
This definition of $P  \left( A \cap B \right)$ in equation \ref{equation:*} can be substituted into the equation defining Conditional Probability (equation \ref{Conditional Probability Definition}) to find Bayes' Theorem (equation \ref{Bayes' Theorem}):

% Equation Conditional Probability Definition
\begin{equation} \tag{1}
		\nonumber P \left( A | B \right) = \frac{ P  \left( A \cap B \right)}{P \left( B \right)}
\end{equation} 

% Equation Bayes' Theorem
\begin{equation} \tag{2}
		\nonumber P \left( A | B \right) = \frac{ P  \left( A  \right) P \left( B | A \right)}{P \left( B \right)}
\end{equation} \\

\clearpage

%
% Solving For (?) using Bayes' Theorem 
%

\section{Question 5}

% 
% Final Row Table 2
%

% Title
\begin{center} Final Row  \end{center}

\begin{center}
%\begin{table}
 \begin{tabular}{||c c c c c||} 
 \hline
 Weather & Temp & Humidity & Windy & Play \\ [0.5ex] 
 \hline\hline
 Sunny & Cool& High & False	& ? \\ [1ex] 
 \hline
\end{tabular}
%\caption{Given Homework 5 Question 5 Table}
%\label{table:kysymys}
%\end{table}
\end{center}

%Getting to equation values from the table entries:
\begin{flalign}
	\hline
	&\, \nonumber \text{        Table To Equation Conversions} \\
	\hline
	&\nonumber \\
	% v
	&\, \nonumber v_{column} = \, \text{selected categorical value of specified column}& \\
	% P(v_{column})
	&\, \nonumber P \left( v_{column} \right) = \frac{\text{number of rows where column entry is $v_{column}$}}{\text{total number of rows}}& \\
	% P(v1|v2)
	&\, \nonumber  P \left( v_{column_1} | v_{column_2} \right) = \frac{\text{number of rows where $column_1$ is $v_{column_1}$ and $column_2$ is $v_{column_2}$}}{\text{number of rows where $column_2$ is $v_{column_2}$}}& \\
	&\nonumber \\
	\hline
	&\nonumber
\end{flalign}


% example v
\iffalse
\begin{equation} 
	\nonumber \text{ex:  } 
	v_{Play} = \, Yes
\end{equation}
\fi

%Substitution Values For final row
\begin{center} Plugging in the given values from final row into equation variables:\end{center}
$v_{Weather} = \, \text{Sunny}$, \\ 
$v_{Temp} = \, \text{Cool}$, \\ 
$v_{Humidity} = \, \text{High}$, \\ 
$v_{Windy} = \, \text{False}$ \\ \\ 
And (in place of the ?) to find the probability that the value in the "Play" column is "Yes"... \\ \\
$v_{Play} = \, \text{Yes}$. \\ \\

% Using Bayes' Theorem for probability of Yes in play column (?)
%Writing Bayes' Theorem to find the probability of a "Yes" value in the "Play" column of the table:


% new command:
\newcommand{\vand}{v_{Weather} \dots \, \cap \, v_{Windy}}
\newcommand{\vprobs}{P \left( v_{Weather}\right) \dots \, \, P \left( v_{Windy} \right) }
\newcommand{\vbarprobs}{P \left( v_{Weather} | v_{Play} \right) \dots \, \, P \left( v_{Windy} | v_{Play} \right) }

%Short hand: \\ \\ $v_{Weather} \cap v_{Temp} \cap v_{Humidity} \cap v_{Windy} = \vand{}$

%
% bayes' theorem direct for question 5
%

% header
\begin{center} Values directly substituted into Bayes' Theorem: \end{center}

% Equation Bayes' Theorem
\begin{equation} \tag{2}
		\nonumber P \left( A | B \right) = \frac{ P  \left( A  \right) P \left( B | A \right)}{P \left( B \right)}
\end{equation}

% direct substitution
\begin{equation}
		\nonumber P \left( v_{Play} | \vand{} \right) 
		= \frac{ P  \left(  v_{Play} \right) P \left( \vand{} |  v_{Play} \right)}{P \left(  \vand{} \right)} 
\end{equation} \\

% shorted notation definition
Where,
\begin{equation}
 \nonumber v_{Weather} \cap v_{Temp} \cap v_{Humidity} \cap v_{Windy} = \vand{}
\end{equation} \\ \\


%
% Assumptions
%

If the columns are independent variables then,
	
	\begin{equation} \tag{3}
		\nonumber P \left( \vand \right) =  \vprobs{}
	\end{equation}
	
	\begin{equation} \tag{4}
		\nonumber P \left( \vand{} |  v_{Play} \right) = \vbarprobs{}
	\end{equation} \\ \\
	
%
% Final equation for problem 5
%
\begin{center} Final Equation \end{center}

Substituting equations 3 and 4 into the version of Bayes' Theorem above:

% bayes' theorem final for question 5
\begin{equation}
		\nonumber P \left( v_{Play} | \vand{} \right) 
		= \frac{ P  \left(  v_{Play} \right) \vbarprobs{}}{ \vprobs{}}
\end{equation} \\

% Result from table
% answer 0.49603174603174593
\begin{center} Results \end{center}
\begin{equation}
		\nonumber P \left( v_{Play} | \vand{} \right) 
		= \frac{ 0.6 * 0.33 * 0.5 * 0.33 * 0.83 }{ 0.4 * 0.5 * 0.4 * 0.7} = 0.49
\end{equation}


\end{document}
